Diese Diplomarbeit dokumentiert die Konzeption, Entwicklung und Umsetzung der mobilen Anwendung WoSamma. Ziel des Projekts war es, ein innovatives, geoinformationsbasiertes Quizspiel zu realisieren, das moderne App-Entwicklung mit spielerischem Lernen verbindet.

Im Mittelpunkt standen dabei sowohl die technische Umsetzung unter Verwendung aktueller Technologien als auch die Gestaltung einer benutzerfreundlichen und interaktiven Anwendung. Neben der Entwicklung der mobilen Applikation wurde besonderes Augenmerk auf die Backend-Architektur, die Datenkommunikation sowie auf soziale Funktionen wie Freundes- und Chatsysteme gelegt.

Die nachfolgenden Kapitel geben einen strukturierten Überblick über die Zielsetzung, die technische Realisierung, das Projektteam sowie die Rahmenbedingungen der Arbeit.
\section{Zusammenfassung}
Im Rahmen dieser Diplomarbeit wurde die mobile Anwendung WoSamma entwickelt – ein geoinformationsbasiertes Quizspiel, das es den Nutzern ermöglicht, Orte innerhalb Österreichs zu erkennen. Die App bietet verschiedene Spielmodi, darunter Einzelspieler-, Freundes- und Mehrspielervarianten. Zusätzlich wurden ein Freundes- und Chatsystem integriert, um den sozialen Aspekt des Spiels zu fördern.
\begin{itemize}
    \item React Native App (mit TypeScript \& Expo)
    \item Supabase (PostgreSQL, Authentifizierung, Storage)
    \item Eigene REST-API für Datenkommunikation
\end{itemize}