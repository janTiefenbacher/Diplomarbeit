\section{Grundlagen der Implementierung}

\subsection{Systemumgebung}
Die entwickelte Anwendung ist ein mobiles Spiel für iOS, umgesetzt mit React Native und TypeScript. 
Für die Entwicklung wurde Visual Studio Code unter macOS verwendet. 
Die Anwendung kann über ein Apple-Developer-Konto direkt auf ein iPhone installiert werden und ist für eine zukünftige Veröffentlichung im Apple App Store vorgesehen.

\subsection{Verwendete Technologien}
\begin{itemize}
    \item \textbf{React Native} – Framework zur Entwicklung plattformübergreifender mobiler Anwendungen.
    \item \textbf{TypeScript} – Typsichere Erweiterung von JavaScript zur Erhöhung der Codequalität.
    \item \textbf{Supabase (PostgreSQL)} – Backend-as-a-Service zur Bereitstellung einer REST-API, Authentifizierung und Datenbank.
    \item \textbf{Google Street View API} – Externe API zur Integration von Street-View-Bildern in das Spiel.
    \item \textbf{GitHub} – Versionsverwaltung und Projektmanagement.
\end{itemize}

\subsection{Architekturüberblick}
Die Anwendung nutzt eine klassische Client–Server-Struktur.  
Der Client (React Native App) kommuniziert über die Supabase-REST-API mit einer PostgreSQL-Datenbank.  
Zusätzlich werden externe Anfragen an die Google Street View API gesendet.

Administrator- und Benutzerrollen sind implementiert, wobei Administratoren erweiterte Berechtigungen besitzen (Freigeben von globalen Nachrichten, Einsehen von Meldungen).

\section{Implementierung der Funktionen}

\subsection{Login und Registrierung}
Die Authentifizierung erfolgt über Supabase Auth.  
Es werden Funktionen zur Registrierung, Anmeldung und Passwortverwaltung bereitgestellt.  
Alle Benutzerdaten werden sicher in der PostgreSQL-Datenbank gespeichert.

\subsection{Benutzerverwaltung}
Das System unterstützt zwei Rollen:
\begin{itemize}
    \item \textbf{Benutzer} – reguläre Spieler.
    \item \textbf{Premium} – reguläre Spieler ohne Werbung.
    \item \textbf{Developer} – erweiterte Rechte, z.\,B. Verwaltung von Meldungen und Veröffentlichung globaler Nachrichten zur Testung der App.
    \item \textbf{Administrator} – erweiterte Rechte, z.\,B. Verwaltung von Meldungen und Veröffentlichung globaler Nachrichten.
\end{itemize}
Spieler können andere melden; Meldungen werden in Supabase gespeichert und durch Administratoren einsehbar.

\subsection{Einzelspieler-Modus}
Der Einzelspielermodus nutzt die Google Street View API zur Anzeige zufälliger Standorte.  
Der Spieler gibt eine Vermutung ab und erhält Punkte basierend auf der Distanz zum tatsächlichen Ort.

\subsection{Mehrspieler-Modus}
Spieler können eine Lobby erstellen, beitreten oder suchen.   
Die Kommunikation erfolgt in Echtzeit über Supabase Channels.  
Jeder Spieler sieht dasselbe Street-View-Bild; das System wertet anschließend die Ergebnisse aus und führt eine Rangliste.

\subsection{Bestenliste}
Die globalen Highscores werden in der Datenbank gespeichert.  
Ein Ranking wird dynamisch aus den Spielergebnissen generiert.

\subsection{Tägliches Spiel}
Jeden Tag steht allen Nutzern dasselbe Bild zur Verfügung.  
Die Ergebnisse werden global verglichen und in einer separaten Tabelle gespeichert.

\subsection{Österreichweite Spielmodi}
Es existieren zwei Varianten:
\begin{itemize}
    \item \textbf{Ganz Österreich} – zufällige Orte im gesamten Bundesgebiet.
    \item \textbf{Spezifische Bundesländer} – Einschränkung auf einzelne Regionen.
\end{itemize}

\subsection{Chat mit Freunden}
Spieler können miteinander chatten.  
Nachrichten werden über Supabase Realtime synchronisiert und persistent gespeichert.

\subsection{Freundesliste und Anfragen}
Es besteht die Möglichkeit:
\begin{itemize}
    \item nach Freunden zu suchen,
    \item Freundschaftsanfragen zu senden und zu akzeptieren,
    \item eine bestehende Liste von Freunden zu verwalten.
\end{itemize}

\subsection{Profilbilder und Shop-System}
Spieler können Profilbilder „kaufen“.  
Der Kaufvorgang ist derzeit simuliert und dient als Platzhalter für ein zukünftiges Zahlungssystem.

\section{Datenhaltung}
Die Datenbank basiert auf PostgreSQL mit Supabase als Service.  
Tabellen umfassen u.\,a.:
\begin{itemize}
    \item Benutzer
    \item Freundesbeziehungen
    \item Chats und Nachrichten
    \item Meldungen
    \item Ergebnisse (Einzelspieler/Mehrspieler)
    \item tägliche Spiele
    \item globale Nachrichten
\end{itemize}

\section{Deployment}
Während der Entwicklungsphase wird die App lokal über ein Apple-Developer-Konto auf iOS-Geräten ausgeführt.  
Eine zukünftige Veröffentlichung im Apple App Store ist vorgesehen.