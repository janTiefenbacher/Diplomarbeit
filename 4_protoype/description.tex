In diesem Kapitel wird die technische Umsetzung des Spiels \textit{WoSamma} dokumentiert.
Dabei werden die verwendete Systemumgebung, eingesetzte Technologien sowie die grundlegenden
Konzepte der Implementierung beschrieben, um einen strukturierten Überblick über die Realisierung
der Anwendung zu geben.

\section{Systemumgebung}
Das Spiel WoSamma ist eine mobile Anwendung für iOS und Android und wurde mit React Native und TypeScript entwickelt.
Die Entwicklung erfolgte unter Windows mithilfe von Visual Studio Code.
Für das Veröffentlichen der Anwendung auf iOS-Geräten sowie für abschließende Tests und Fehlerbehebungen wurde Xcode unter macOS in Verbindung mit einem Apple-Developer-Konto verwendet.
Die Tests für Android wurden mit Android Studio auf Emulatoren durchgeführt.
Die Anwendung ist für eine zukünftige Veröffentlichung im Apple App Store sowie im Google Play Store vorgesehen.

\subsection{Verwendete Technologien}
\begin{itemize}
    \item \textbf{React Native} – Framework zur Entwicklung plattformübergreifender mobiler Anwendungen für iOS und Android.
    \item \textbf{TypeScript} – Typsichere Erweiterung von JavaScript, die bewusst anstelle von JavaScript gewählt wurde, um die Wartbarkeit, Lesbarkeit und Codequalität der Anwendung zu erhöhen.
    \item \textbf{Supabase (PostgreSQL)} – Backend-as-a-Service, das zentrale Funktionen wie Authentifizierung, Echtzeitdaten, Datenbankverwaltung sowie eine REST-API bereitstellt und damit den Entwicklungsaufwand deutlich reduziert.
    \item \textbf{Google Maps API} – API zur Verarbeitung und Darstellung geografischer Daten sowie zur Positionsbestimmung innerhalb des Spiels.
    \item \textbf{Google Street View API} – API zur Integration von 360-Grad-Street-View-Bildern aus ganz Österreich, welche die Grundlage für das GeoGuessr-ähnliche Spielkonzept bildet.
    \item \textbf{GitHub} – Plattform zur Versionsverwaltung, die eine strukturierte Zusammenarbeit im Team sowie die kontinuierliche Aktualität des Quellcodes sicherstellt.
    \item \textbf{Jira} – Tool zur Projektplanung und Aufgabenverwaltung, das im Rahmen einer agilen Vorgehensweise für die Diplomarbeit eingesetzt wurde.
    \item \textbf{Figma} – Design- und Prototyping-Tool zur Erstellung von Mockups und interaktiven Prototypen, welche als visuelle Grundlage für die Benutzeroberfläche dienten.
\end{itemize}

\section{Implementierung}
Dieses Kapitel beschreibt die technische Umsetzung des Spiels \textit{WoSamma}. 
Dabei wird auf den Aufbau der Anwendungsarchitektur, die Implementierung des Frontends, 
die Anbindung des Backends sowie auf die zentrale Spiellogik eingegangen. 
Der Fokus liegt auf den verwendeten Konzepten, der Struktur des Quellcodes 
und den getroffenen technischen Entscheidungen.

\subsection{Architektur der Anwendung}
Die Anwendung \textit{WoSamma} folgt einer klaren Trennung zwischen Frontend, Backend 
und externen Diensten. 
Das Frontend wurde mit React Native umgesetzt und ist für die Benutzeroberfläche, 
die Spiellogik sowie die Interaktion mit dem Benutzer verantwortlich.
Das Backend basiert auf Supabase und übernimmt Aufgaben wie Authentifizierung, 
Datenpersistenz und Echtzeitkommunikation.
Externe APIs, wie die Google Maps und Google Street View API, werden zur Bereitstellung 
von geografischen Daten und 360-Grad-Bildern verwendet.


GRAFIKK!!!!!!!!!

\subsection{Frontend-Implementierung - JAN NOCH UMSCHREIBEN}
In diesem Abschnitt wird die Umsetzung der Benutzeroberfläche von \textit{WoSamma} beschrieben.
Der Fokus liegt auf der Struktur der Anwendung, der Navigation zwischen verschiedenen Bildschirmen
sowie der Interaktion des Benutzers mit der Spielansicht.
Besonderes Augenmerk liegt dabei auf der Darstellung von Spielinformationen und
der Visualisierung von Spielfortschritten sowohl im Einzelspieler- als auch im Mehrspielermodus.

\subsubsection{Projektstruktur und Navigation}
Die Projektstruktur orientiert sich an bewährten Konzepten von React Native und
TypeScript, um eine modulare, wiederverwendbare und wartbare Codebasis zu gewährleisten.
Die Navigation zwischen den verschiedenen Bildschirmen, wie Startbildschirm, Spielansicht,
Freundeslobby und Profilen, erfolgt über ein Stack- und Tab-basiertes Navigationssystem.
Dieses erlaubt einen intuitiven Zugriff auf alle relevanten Funktionen und stellt sicher,
dass der Benutzer jederzeit den Überblick über den Spielstatus behält.

\subsubsection{Spielansicht und Benutzerinteraktion}
Die Spielansicht bildet das zentrale Interface für den Spieler und ist in Einzelspieler-
und Mehrspielermodus unterteilt.

\paragraph{Einzelspieler-Modus:}
Im Einzelspieler-Modus wird dem Benutzer die korrekte Location sowie die vom Spieler
geratene Position angezeigt.
Die Distanz zwischen der tatsächlichen Location und der gewählten Position wird berechnet
und visualisiert, wodurch der Spieler unmittelbar Feedback zu seiner Genauigkeit erhält.
Zusätzlich werden die Punkte basierend auf dieser Distanz angezeigt, um den Fortschritt
und die Leistung nachvollziehbar zu machen.

\paragraph{Mehrspieler-Modus:}
Im Mehrspieler-Modus werden alle Spieler in einer Lobby zusammengeführt.
Jeder Spieler setzt seine Pins auf der Karte, welche live an alle Teilnehmer übertragen werden.
Zusätzlich wird der richtige Pin angezeigt, um die Vergleichbarkeit der Ergebnisse zu gewährleisten.
Ein Vektor zwischen der geratenen Position jedes Spielers und dem tatsächlichen Standort
wird dargestellt, und die Distanz sowie die erreichten Punkte werden für alle Spieler
übersichtlich angezeigt.
Dies ermöglicht ein direktes, kompetitives Spielerlebnis und fördert die Interaktion
zwischen den Teilnehmern.


\subsection{Backend-Anbindung mit Supabase}
\subsubsection{Datenbankstruktur}
\subsubsection{Authentifizierung der Benutzer - Supabase Auth, OAuth2}

\subsection{Spielmechanik und Logik}
Dieser Abschnitt beschreibt die grundlegende Spielmechanik von \textit{WoSamma}.
Dabei wird erläutert, wie Spielrunden erstellt werden, wie Mehrspieler-Partien
mit Freunden umgesetzt sind und nach welchen Kriterien die Punktevergabe erfolgt.

\subsubsection{Generierung der Spielrunden}
\subsubsection{Generierung des täglichen Spiels}
\subsubsection{Live-Einladungen für Freundesrunden}
\subsubsection{Punkteberechnung}

\subsection{Benutzerprofil}
Dieser Abschnitt beschreibt die Umsetzung des Benutzerprofils innerhalb der Anwendung.
Das Benutzerprofil dient zur Darstellung spielrelevanter Informationen sowie zur Verwaltung
von freigeschalteten Inhalten und individuellen Einstellungen.
\subsubsection{Profilanpassung}
\subsubsection{Kaufen und Freischalten von Profilinhalten}

\subsection{Freundesystem}
Das Freundesystem ermöglicht die soziale Interaktion zwischen den Benutzern der Anwendung.
Es dient dazu, Spieler miteinander zu vernetzen, gemeinsame Spielrunden zu starten
und den Austausch innerhalb der Anwendung zu fördern.
\subsubsection{Freundschaftsanfragen}
\subsubsection{Herausfordern von Freunden}
\subsubsection{Chatfunktion}
\subsubsection{Anzeigen von Benutzerprofilen}


\subsection{Integration der Google APIs}
\subsubsection{Verwendung der Google Maps API}
\subsubsection{Einbindung der Street-View-Bilder}
