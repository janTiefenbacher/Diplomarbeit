\section{Spezifische Ausgangslage}\label{chapter:loesung1}
\subsection{Spezifische Ausgangslage - Jan Tiefenbacher}
\subsection{Spezifische Ausgangslage – Laurenz Pichler}

Die zentrale Aufgabe unserer Diplomarbeit liegt darin, eine mobile Applikation zu entwickeln, welche auf mobilen iOS- sowie Android-Geräten lauffähig ist. Wir sahen uns zu Beginn unserer Arbeit mit einer Entscheidung konfrontiert. Es ging um die Fragestellung, welches Framework wir für die Realisierung unseres Projekts verwenden.

Für die inhaltliche Ausgestaltung der Applikation ließen wir uns von dem populären Spiel „GeoGuessr“ inspirieren. Obschon wir dieses Spielkonzept als sehr reizvoll empfanden, offenbarte sich rasch, dass es überaus herausfordernd ist, globale, zufällige Orte präzise zu identifizieren und zu erraten, insbesondere für Nutzer ohne geographisches Fachwissen. Wir konzipieren die Idee zu „WoSamma“, um dieses Prinzip weiterhin erfolgreich zu realisieren und es zugleich an unsere Projektgegebenheiten anzupassen.


\section{Spezifische Forschungsfrage}
\subsection{Spezifische Forschungsfrage - Jan Tiefenbacher}
\subsection{Spezifische Forschungsfrage - Laurenz Pichler}
Welche technischen und organisatorischen Skalierungsstrategien ermöglichen es, eine bestehende App-Infrastruktur kurzfristig und langfristig an stark wachsende Nutzerzahlen anzupassen?

