\section{Spezifische Ausgangslage }\label{chapter:loesung1}

Die zentrale Aufgabe dieser Diplomarbeit besteht in der Konzeption und Entwicklung einer mobilen Applikation, die sowohl auf iOS- als auch auf Android-Endgeräten lauffähig ist. Bereits zu Beginn des Projekts ergab sich eine grundlegende technische Fragestellung hinsichtlich der Wahl eines geeigneten Frameworks, das eine plattformübergreifende Entwicklung ermöglicht, gleichzeitig jedoch den Anforderungen an Performance, Wartbarkeit und Erweiterbarkeit gerecht wird. Diese Entscheidung stellte einen wesentlichen Einflussfaktor auf die gesamte weitere Projektumsetzung dar.

Für die inhaltliche Ausgestaltung der Applikation diente das bekannte Online-Spiel GeoGuessr als konzeptionelle Inspiration. Dieses Spielprinzip basiert auf der zufälligen Platzierung von Spieler an realen Orten, die anhand visueller Hinweise identifiziert werden müssen. Im Zuge erster Analysen zeigte sich jedoch, dass die globale Ausrichtung dieses Konzepts für viele Nutzer eine erhebliche Einstiegshürde darstellt. Insbesondere Personen ohne ausgeprägtes geografisches Fachwissen stoßen rasch an ihre Grenzen, was sowohl den Spielspaß als auch die langfristige Nutzung beeinträchtigen kann.

Aus dieser Beobachtung heraus entstand die Idee zur Entwicklung von WoSamma, einer Applikation, die das bewährte Spielprinzip aufgreift, dieses jedoch gezielt auf Österreich beschränkt. Durch die regionale Eingrenzung soll einerseits die Zugänglichkeit erhöht und andererseits ein spielerischer Lernmehrwert geschaffen werden. Gleichzeitig eröffnet dieses klar definierte Zielgebiet neue Möglichkeiten hinsichtlich der inhaltlichen Ausgestaltung, Nutzerbindung sowie wirtschaftlichen Verwertung der Anwendung.

Im Zuge der Projektplanung rückten neben der technischen Realisierung zunehmend auch wirtschaftliche und strukturelle Fragestellungen in den Fokus. Insbesondere stellte sich die Frage, wie eine solche Anwendung langfristig betrieben werden kann, ohne die Nutzererfahrung negativ zu beeinflussen, sowie wie die zugrunde liegende Systemarchitektur gestaltet sein muss, um auch bei steigenden Nutzerzahlen stabil und performant zu bleiben. Diese Überlegungen bilden die Grundlage für die im weiteren Verlauf der Arbeit behandelten spezifischen Forschungsfragen.

\section{Spezifische Forschungsfragen}
Um die Entwicklung der Anwendung strukturiert und zielgerichtet durchführen zu können, wurden im Vorfeld zentrale Forschungsfragen definiert. Diese dienen als fachliche und technische Leitlinie für die Umsetzung des Projekts.

Die Forschungsfragen beziehen sich insbesondere auf die effiziente Realisierung einer plattformübergreifenden mobilen Anwendung, die Integration geobasierter Funktionen sowie die Gestaltung einer skalierbaren und sicheren Backend-Infrastruktur.

Im Folgenden werden diese Fragestellungen dargestellt und im weiteren Verlauf der Arbeit systematisch beantwortet.
\subsection{Spezifische Forschungsfrage - Jan Tiefenbacher}
Welche Monetarisierungsmodelle sind für diese App am effektivsten, um eine nachhaltige Einnahmengenerierung zu ermöglichen und gleichzeitig eine ausgewogene Balance zwischen Wirtschaftlichkeit und Nutzerzufriedenheit zu gewährleisten?
\subsection{Spezifische Forschungsfrage - Laurenz Pichler}
Welche technischen und organisatorischen Skalierungsstrategien ermöglichen es, eine bestehende App-Infrastruktur kurzfristig und langfristig an stark wachsende Nutzerzahlen anzupassen?