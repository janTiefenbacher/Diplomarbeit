Die fortschreitende Digitalisierung sowie die zunehmende Verbreitung mobiler Endgeräte eröffnen neue Möglichkeiten im Bereich interaktiver Anwendungen. Insbesondere spielerische Lernkonzepte gewinnen an Bedeutung, da sie Wissensvermittlung mit Motivation und Unterhaltung verbinden.
\section{Ausgangslage und Motivation der Arbeit}
Die Motivation für dieses Projekt entstand aus dem Wunsch, moderne App-Entwicklung mit einem spielerischen und zugleich lehrreichen Ansatz zu verbinden. Geografisches Wissen über Österreich ist im Alltag häufig nur oberflächlich vorhanden, obwohl es ein wesentlicher Bestandteil kultureller und regionaler Identität ist.

Ziel war es daher, eine Anwendung zu entwickeln, die geografisches Wissen auf interaktive Weise vermittelt und gleichzeitig durch verschiedene Spielmodi und soziale Funktionen langfristig motiviert.

Darüber hinaus bot das Projekt die Möglichkeit, praktische Erfahrungen in der Full-Stack-Entwicklung zu sammeln – von der mobilen Benutzeroberfläche über Authentifizierungssysteme bis hin zur Entwicklung einer eigenen REST-API. Die Kombination aus technischer Herausforderung, praxisnaher Umsetzung und kreativem Gestaltungsspielraum stellte einen besonderen Anreiz dar.

Als Inspiration diente das erfolgreiche Spielkonzept von "GeoGuessr", das weltweit Millionen von Nutzern begeistert. Die Idee, dieses Konzept auf Österreich zu fokussieren und um soziale Interaktionen zu erweitern, bildete die Grundlage für die Entwicklung von WoSamma.