In den Theoretischen Grundlagen werden die zentralen Grundlagen für die Entwicklung der App erläutert.
\section{App-Design \& Monetarisierungskonzepte}
Im ersten Teil der Theoretischen Grundlagen werden das Design und die möglichen Monetarisierungskonzepte
mobiler Applikationen erläutert. 
% --------------------------------------------------
\subsection{Plattformwahl \& technische Grundlagen}

Wenn man sich dazu entscheidet eine mobile Applikation zu entwickeln, muss man sich zuerst die Frage stellen,
auf welchen Plattformen die App laufen soll. Die beiden dominierenden Betriebssysteme im mobilen App Bereich sind 
iOS von Apple und Android von Google. Diese Entscheidung hat nicht nur Auswirkungen auf die technische Umsetzung,
sondern auch auf das Design, die Verfügbarkeit und die Monetarisierungsmöglichkeiten der App. 
\vspace{0.5em}

\textbf{iOS}\\
iOS ist das mobile Betriebssystem von Apple, das die technologische Grundlage für Geräte wie das iPhone und iPad bildet. 
Es ist speziell für Touch-Bedienung und intuitive Nutzung konzipiert und bekannt für hohe Sicherheitsstandards, eine intuitive 
Benutzeroberfläche und die nahtlose Integration ins Apple-Ökosystem. Technisch basiert iOS auf einem Unix-ähnlichen System-Kernel 
namens Darwin, was eine solide, stabile Basis für moderne mobile Anwendungen schafft. Da Apple Hard- und Software eng verzahnt und 
die Plattform stark kontrolliert, können Updates, Sicherheitsmechanismen und Apple-Dienste über alle unterstützten Geräte sehr einheitlich 
bereitgestellt werden. Um eine Applikation für iOS Geräte verfügbar zu machen, muss diese über den Apple App Store vertrieben werden,
wobei strenge Richtlinien und Prüfprozesse sicherstellen, dass nur qualitativ hochwertige und sichere Apps zugelassen werden. Außerdem braucht man 
eine Apple Developer Lizenz, um Apps im App Store veröffentlichen zu können.\cite{ComputerWeekly_Apple_iOS} \cite{IT_Schulungen_Was_ist_iOS} \cite{Ionos_IOS_App_Entwicklung} 

\vspace{0.5em}
\textbf{Android}\\
Android ist ein Linux-basiertes, mobiles Betriebssystem von Google, das hauptsächlich auf Smartphones und Tablets läuft und als Plattform 
alle System- und Benutzerkomponenten umfasst, also das Linux-Kernel-Betriebssystem, die grafische Oberfläche und die nutzbaren Apps. Android 
wurde unter der Apache-Open-Source-Lizenz veröffentlicht, was Herstellern und Entwicklern erlaubt, die Software anzupassen oder eigene Varianten 
zu bauen. Obwohl die Basis offen ist, enthalten die meisten Geräte zusätzliche proprietäre Programme wie Google-Apps, die vorinstalliert sind. 
Dadurch ist Android heute das am weitesten verbreitete mobile Betriebssystem weltweit. Um eine Applikation für Android-Geräte verfügbar zu machen, 
wird diese in der Regel über den Google Play Store vertrieben. Auch hier gibt es Richtlinien und Prüfverfahren, die sicherstellen sollen, dass nur 
funktionierende und sichere Apps veröffentlicht werden. Zusätzlich benötigt man ein Google Developer-Konto, um Anwendungen im Play Store 
bereitstellen zu können. \cite{Featured_Android} \cite{IONOS_Android_App_Entwicklung}

\vspace{0.5em}
\textbf{Plattformspezifische Unterschiede zwischen iOS und Android}\\
iOS- und Android-Systeme unterscheiden sich unter anderem in ihrer Systemarchitektur, 
Designrichtlinien und Gerätevielfalt. Während iOS auf eine begrenzte Anzahl an Endgeräten 
optimiert ist, existiert im Android-Bereich eine große Vielfalt an Bildschirmgrößen und 
Hardwarekonfigurationen. Diese Unterschiede erhöhen den Entwicklungs- und Wartungsaufwand 
bei nativen Anwendungen.

\vspace{0.5em}
\textbf{Native Apps – Technische Grundlagen}\\
Bei der nativen App-Entwicklung handelt es sich um die Entwicklung von Anwendungen, die speziell für ein bestimmtes Betriebssystem programmiert werden. Für iOS-Anwendungen wird beispielsweise die Programmiersprache Swift verwendet, während Android-Applikationen typischerweise in Kotlin entwickelt werden. Die Anwendung wird direkt für das jeweilige Betriebssystem entwickelt und ist nicht mit anderen Systemen kompatibel.

Ein wesentliches Merkmal nativer Anwendungen ist der direkte Zugriff auf systemnahe Funktionen und Hardware-Komponenten des Endgeräts, wie Kamera, GPS oder Sensoren. Native Apps können zudem systeminterne Funktionen wie Push-Benachrichtigungen nutzen. Durch die enge Integration mit dem Betriebssystem ist eine optimale Nutzung von Ressourcen wie Arbeitsspeicher und Hardware möglich, was zu einer guten Performance und hoher Usability führt.

Allerdings erfordert dieser Ansatz für jedes Betriebssystem eine eigene Implementierung, wodurch separate Versionen für iOS und Android entwickelt werden müssen. Dies erhöht den Entwicklungsaufwand deutlich. \cite{Denvo_Native_vs_Crossplattform} \cite{ITPortal_Native_vs_Crossplattform} \cite{knguru_crossplatform}

\vspace{0.5em}
\textbf{Native Apps – Design und Monetarisierung}\\
Native Apps orientieren sich stark an den Design- und Interaktionsrichtlinien des jeweiligen Betriebssystems. Dadurch fügen sie sich nahtlos in die Benutzeroberfläche von iOS oder Android ein und bieten eine konsistente User Experience. Die hohe Usability entsteht durch die Verwendung systemeigener UI-Elemente und Interaktionsmuster, die den Nutzerinnen und Nutzern bereits vertraut sind.

Die Bereitstellung nativer Apps erfolgt über die jeweiligen App Stores der Betriebssysteme. Dadurch sind sie direkt an das Ökosystem der Plattform gebunden und können die dort vorgesehenen Mechanismen zur Verteilung und Nutzung verwenden. \cite{Denvo_Native_vs_Crossplattform} \cite{ITPortal_Native_vs_Crossplattform} \cite{knguru_crossplatform}

\vspace{0.5em}
\textbf{Cross-Platform Apps – Technische Grundlagen}\\
Bei der Cross-Platform-App-Entwicklung wird eine gemeinsame Codebasis verwendet, um Anwendungen für mehrere Betriebssysteme wie iOS und Android zu realisieren. Mithilfe spezieller Frameworks wie Flutter, React Native oder Xamarin wird dieser Code für die jeweiligen Plattformen umgesetzt. Ziel dieses Ansatzes ist es, große Teile des Codes plattformübergreifend wiederzuverwenden.

Cross-Platform-Frameworks abstrahieren die zugrunde liegenden Betriebssysteme und stellen einheitliche Schnittstellen zur Verfügung. Dadurch ist es möglich, eine Anwendung zu entwickeln, ohne für jedes Betriebssystem eine vollständig eigene Implementierung zu erstellen. Die Interaktion mit gerätespezifischen Funktionen erfolgt dabei häufig über zusätzliche Abstraktionsschichten oder spezielle Erweiterungen.

Durch die gemeinsame Codebasis verringert sich der Entwicklungsaufwand, und Änderungen müssen nur einmal durchgeführt werden, um sie auf allen unterstützten Plattformen verfügbar zu machen. Dies erleichtert zudem die Wartung der Anwendung. \cite{Denvo_Native_vs_Crossplattform} \cite{ITPortal_Native_vs_Crossplattform} \cite{knguru_crossplatform}

\vspace{0.5em}
\textbf{Cross-Platform Apps – Design und Monetarisierung}\\
Cross-Platform-Apps können viele native Funktionen nutzen und sich in vielerlei Hinsicht wie native Anwendungen anfühlen, auch wenn sie nicht vollständig plattformspezifisch entwickelt wurden. Das Design ist dabei weniger strikt an die Richtlinien eines einzelnen Betriebssystems gebunden, da eine einheitliche Umsetzung für mehrere Plattformen angestrebt wird.

Durch die schnellere Markteinführung und den geringeren Entwicklungsaufwand eignet sich dieser Ansatz besonders für Anwendungen, die gleichzeitig auf mehreren Plattformen verfügbar sein sollen. \cite{Denvo_Native_vs_Crossplattform} \cite{ITPortal_Native_vs_Crossplattform} \cite{knguru_crossplatform}


\vspace{0.5em}
\textbf{Cross-Platform-Lösung}\\
Bevor mit der eigentlichen Entwicklung der App begonnen werden konnte, musste zunächst eine Entscheidung 
bezüglich der Zielplattform getroffen werden. Aufgrund der definierten Zielgruppe sowie der angestrebten 
Reichweite fiel die Wahl auf eine Cross-Platform-Lösung, um sowohl iOS- als auch Android-Nutzer zu erreichen.

\vspace{0.5em}
\textbf{Einsatz von React Native}\\
Für die Umsetzung des Frontends wurde das Framework React Native eingesetzt. 
React Native ist ein von Facebook entwickeltes Framework, das es erlaubt, mobile Anwendungen unter 
Verwendung von JavaScript und React zu erstellen und eine gemeinsame Codebasis für iOS und Android zu nutzen. 
Dabei werden Benutzeroberflächen aus modularen, wiederverwendbaren Komponenten aufgebaut, die jeweils 
eigenständige UI-Elemente wie Schaltflächen oder Ansichten repräsentieren und so eine klare Struktur der 
App gewährleisten. Durch dieses komponentenbasierte System können Entwickelnde plattformspezifische 
Unterschiede adressieren und zugleich einen einheitlichen Look \& Feel über beide Zielplattformen erreichen, 
da React Native die native Darstellung der Komponenten auf iOS und Android übernimmt. React Native bietet 
zudem die Möglichkeit, bei Bedarf auf native APIs zuzugreifen, um plattformspezifische Funktionalität zu 
integrieren. \cite{Denvo_Native_vs_Crossplattform} \cite{ITPortal_Native_vs_Crossplattform} \cite{knguru_crossplatform}


% --------------------------------------------------
\subsection{Trends \& Weiterentwicklungen im App-Design}

\vspace{0.5em}
\textbf{Modernes App Design}\\
Im Jahr 2025 hat sich Design im digitalen Kontext von einer rein visuellen Disziplin zu einem zentralen strategischen 
Faktor entwickelt. In einem stark gesättigten Markt mit einer Vielzahl an Apps und digitalen Services entscheidet nicht 
mehr allein die Funktionalität über den Erfolg eines Produkts, sondern vor allem die Qualität der User Experience. 
Nutzer erwarten intuitive, schnelle und personalisierte Anwendungen, die sich nahtlos in ihren Alltag integrieren. 
Design beeinflusst dabei direkt Nutzerbindung, Verweildauer und Akzeptanz digitaler Produkte.

\vspace{0.5em}
\textbf{Dark und Light Mode}\\
Der Dark Mode hat sich in der modernen Web- und App-Entwicklung längst als Standard etabliert und ist nicht mehr nur eine 
optionale Designentscheidung. Während früher der Light Mode als Marktstandard galt und der Dark Mode lediglich als Zusatzfunktion 
angeboten wurde, hat sich dieses Verhältnis in den letzten Jahren komplett gewandelt. Heute setzen die meisten mobilen Anwendungen 
standardmäßig auf den Dark Mode und bieten, wenn überhaupt, einen optionalen Light Mode an. Neben ästhetischen Aspekten überzeugt 
der Dark Mode vor allem durch ergonomische Vorteile wie eine reduzierte Augenbelastung, insbesondere in dunklen Umgebungen, sowie 
potenzielle Energieeinsparungen auf OLED- und AMOLED-Displays. Moderne Designs berücksichtigen daher unterschiedliche Lichtverhältnisse 
und Nutzungskontexte und passen Kontraste sowie Farbschemata dynamisch an, um sowohl Nutzbarkeit als auch Zugänglichkeit zu optimieren. 
Dennoch bleibt der Light Mode in hellen Umgebungen oder für bestimmte Nutzergruppen weiterhin relevant, weshalb eine flexible Umschaltmöglichkeit 
zwischen beiden Modi als bewährte Praxis gilt. Genau aus diesem Grund stellen Dark Mode und Light Mode gemeinsam einen wichtigen Bestandteil 
einer zeitgemäßen Designstrategie dar. \cite{Publizer_dark_mode_light_mode} \cite{natively_mobile_app_design_trends_2025}


\vspace{0.5em}
\textbf{Responsive \& Adaptive Design}\\
Responsive und Adaptive Design beschreibt zwei zentrale Ansätze moderner Web und App Gestaltung, die als Reaktion auf die starke Diversifizierung 
internetfähiger Endgeräte entstanden sind. Während vor dem mobilen Web weitgehend homogene Bildschirmgrößen dominierten, müssen Anwendungen heute 
auf Displaybreiten von etwa 320 Pixel bis über 4.000 Pixel sowie unterschiedliche Eingabemethoden und Auflösungen reagieren. Responsive Design 
verfolgt dabei einen flexiblen, fließenden Ansatz, bei dem sich ein einziges Layout mithilfe relativer Einheiten, CSS Media Queries, moderner 
Layout-Module wie Flexbox oder Grid sowie Techniken wie Mobile First dynamisch an den verfügbaren Bildschirmplatz anpasst, um eine konsistente 
User Experience auf allen Geräten zu gewährleisten. Adaptive Design hingegen arbeitet mit mehreren vordefinierten, eher starren Layouts, die 
abhängig von Geräteeigenschaften wie Bildschirmgröße oder Ausrichtung geladen werden und häufig feste Pixelwerte verwenden. Beide Ansätze zielen 
darauf ab, Usability, Performance und Designqualität zu optimieren, unterscheiden sich jedoch in ihrer Philosophie: Während Responsive Design das 
Verhalten der Inhalte definiert, legt Adaptive Design das konkrete Darstellungsergebnis für bestimmte Gerätekategorien fest. In der Praxis werden 
die Vorteile beider Konzepte oft kombiniert, um sowohl Flexibilität als auch gezielte Optimierung für ausgewählte Endgeräte zu erreichen. \cite{Ionos_responsive_design} \cite{kinsta_responsive_vs_adaptive}

\vspace{0.5em}
\textbf{Accessibility \& Barrierefreiheit}\\
Accessibility bzw. Barrierefreiheit beschreibt das Ziel, digitale Produkte so zu gestalten und umzusetzen, dass sie von möglichst allen Menschen 
gleichwertig genutzt werden können. Das betrifft nicht nur Personen mit dauerhaften Einschränkungen wie Seh- oder Hörbehinderungen, motorischen oder 
kognitiven Beeinträchtigungen, sondern auch situative Einschränkungen, etwa wenn jemand kurzfristig eine verletzte Hand hat oder bei starker Sonne kaum 
etwas am Display erkennt. Barrierefreiheit ist damit kein „Extra-Feature“, sondern ein Qualitätsmerkmal guter User Interfaces: Wenn eine App klar strukturiert, gut 
bedienbar und verständlich ist, profitieren am Ende alle Nutzer. \cite{uxmatters} \cite{materialdesign}

\vspace{0.5em}
\textbf{Warum Barrierefreiheit wichtig ist}\\
Barrierefreiheit hat mehrere Ebenen an Relevanz. Aus ethischer Sicht geht es um digitale Teilhabe, Apps und Websites sind heute zentrale Zugänge zu Information, 
Kommunikation und Services, weshalb Ausschlüsse durch schlechtes Design reale Konsequenzen haben. Zusätzlich spielt eine rechtliche Dimension hinein, 
da Accessibility-Anforderungen in vielen Ländern und Branchen stärker reguliert werden. Und auch wirtschaftlich ist das Thema relevant. Barrierefreie 
Produkte verbessern oft die Markenwahrnehmung, erhöhen die Nutzerbindung und erschließen zusätzliche Zielgruppen, statt potenzielle User einfach 
auszulassen.\cite{uxmatters} \cite{materialdesign}

\vspace{0.5em}
\textbf{Struktur, Hierarchie und Fokusführung}\\
Ein Kernpunkt barrierefreier Gestaltung ist eine klare Informationshierarchie. Nutzer müssen schnell erkennen können, wo sie sind und was die wichtigsten 
Aktionen sind. Dabei ist nicht nur das visuelle Layout relevant, sondern auch die Reihenfolge, in der Inhalte technisch angeordnet sind. Screenreader lesen 
Inhalte typischerweise in einer top-down Reihenfolge aus dem Markup. Deshalb ist die Zusammenarbeit zwischen Design und Development wichtig, damit visuelle 
Hierarchie, DOM-Reihenfolge und Fokuslogik zusammenpassen. Elemente sollen
in einer logischen Reihenfolge erreichbar sein und Gruppierungen sollen verständlich sein, damit Nutzer sich schnell orientieren können. \cite{uxmatters} \cite{materialdesign}

\vspace{0.5em}

\textbf{Wahrnehmbarkeit: Kontrast, Farbe und Typografie}\\
Damit Inhalte für möglichst viele Menschen wahrnehmbar sind, spielen Kontrast und Lesbarkeit eine zentrale Rolle. Ausreichende Kontraste zwischen Text 
und Hintergrund helfen insbesondere bei Sehbehinderungen, aber auch in Alltagssituationen wie starkem Umgebungslicht. Wichtig ist außerdem, Informationen 
nicht ausschließlich über Farbe zu kommunizieren, beispielsweise einen Fehler nur Rot zu markieren, sondern zusätzliche Hinweise wie Text, Icons oder Umrandungen zu 
verwenden. Typografie und Layout sollten so angelegt sein, dass größere Schriftgrößen und Zoom nicht zu überlappenden oder abgeschnittenen Elementen führen. 
Flexible Layouts, ausreichende Abstände und skalierbare Schriftgrößen sind hier zentrale Bausteine.\cite{uxmatters} \cite{materialdesign}

\vspace{0.5em}
\textbf{Bedienbarkeit: Touch Targets und Eingabemethoden}\\
Gerade auf mobilen Geräten ist die Größe von Interaktionsflächen entscheidend. Kleine Icons ohne ausreichende Abstände führen schnell zu Fehlbedienungen, 
besonders bei motorischen Einschränkungen. Deshalb sollten Buttons, Icons und interaktive Elemente genügend 
große Touch- bzw. Pointer-Flächen haben und mit ausreichendem Abstand zueinander platziert werden.\cite{uxmatters} \cite{materialdesign}

\vspace{0.5em}
\textbf{Labels und Alt-Text}\\
Ein weiterer zentraler Baustein ist eine klare und aussagekräftige Textgestaltung innerhalb der Anwendung. Dazu zählen sichtbare Beschriftungen wie Buttontexte sowie kurze beschreibende Texte für Symbole und grafische Elemente. Diese Inhalte sollten präzise, eindeutig und handlungsorientiert formuliert sein, um eine schnelle Orientierung und effiziente Nutzung zu ermöglichen.

Auch bei Bildern ist eine bewusste Textzuordnung wichtig, sofern sie relevante Informationen vermitteln. Der Text sollte sich dabei auf das Wesentliche beschränken und den inhaltlichen Zweck des Bildes erklären. Grafische Elemente ohne Informationsgehalt oder mit bereits erklärendem Kontext können bewusst ohne zusätzliche Beschreibung eingesetzt werden, um Wiederholungen zu vermeiden.\cite{uxmatters} \cite{materialdesign}

\vspace{0.5em}
\textbf{Testing und Umsetzung}\\
Barrierefreiheit entsteht nicht nur durch „Guidelines lesen“, sondern durch konsequentes Testen. Standard-UI-Komponenten und semantisches Markup sind oft eine 
stabile Basis, während Custom Widgets schnell mehr Aufwand und Fehlerquellen bringen. In der Praxis sollten zentrale Tasks end-to-end mit aktivierten 
Accessibility-Funktionen getestet werden, wie Screenreader-Navigation, Fokus-Reihenfolge, große Schrift und Kontrastanpassungen. Zusätzlich sind Tests mit 
betroffenen Nutzer:innen extrem wertvoll, weil sie reale Nutzungsmuster sichtbar machen, die im Team sonst leicht übersehen werden.



% --------------------------------------------------
\vspace{0.5em}

\subsection{UI/UX-Design \& Nutzererlebnis}

\textbf{Zielgruppenanalyse}\\
Eine zielgerichtete Kommunikation und ein passgenaues Angebot setzen voraus, dass klar ist, wer überhaupt angesprochen werden soll und warum
diese Personen ein Produkt oder eine Dienstleistung nutzen würden. Genau hier setzt die Zielgruppenanalyse an. Sie ist ein strukturierter Prozess,
um potenzielle und bestehende Nutzer besser zu verstehen und daraus konkrete Entscheidungen für Content, Design, Funktionen,
Marketing und Produktstrategie abzuleiten. In digitalen Projekten bildet sie damit eine zentrale
Grundlage für nutzerorientierte Entwicklung und eine konsistente Markenkommunikation. \cite{appinio_zielgruppenanalyse} \cite{webdesign_journal_zielgruppenanalyse}\\

\textbf{Begriff und Bedeutung der Zielgruppe}\\
Eine Zielgruppe beschreibt eine Gruppe von Menschen, die hinsichtlich bestimmter Merkmale wie Alter, Bedürfnisse, Verhalten und
Budget ausreichend ähnlich sind, um mit gezielten Maßnahmen angesprochen werden zu können. Die Zielgruppe umfasst dabei nicht nur bereits bestehende
Kunden, sondern auch potenzielle Nutzer, die prinzipiell Interesse am Angebot haben könnten. \cite{appinio_zielgruppenanalyse} \cite{webdesign_journal_zielgruppenanalyse}\\

\textbf{Ziel und Inhalt der Zielgruppenanalyse}\\
Die Zielgruppenanalyse umfasst alle Aktivitäten, die dazu dienen, diese Gruppe genauer zu beschreiben. Dabei geht es nicht nur um „harte Fakten“
wie demografische Daten, sondern vor allem um die Frage: Welche Bedürfnisse, Motive, Barrieren und Erwartungen bestimmen Entscheidungen und
Verhalten? Ziel ist es, ein belastbares Verständnis zu gewinnen, um Maßnahmen nicht nach Bauchgefühl zu planen, sondern zielgerichtet auszurichten.\cite{appinio_zielgruppenanalyse} \cite{webdesign_journal_zielgruppenanalyse}\\

\textbf{Risiken fehlender Zielgruppenanalyse}\\
Ohne Zielgruppenanalyse besteht das Risiko, dass Produkte, Inhalte oder Kommunikationsmaßnahmen an den Bedürfnissen der Nutzer vorbeigehen. Das führt
häufig zu ineffizientem Ressourceneinsatz, geringer Reichweite, schwacher Conversion und langfristig zu sinkender Kundenzufriedenheit. Umgekehrt
ermöglicht eine fundierte Analyse, Angebote und Inhalte zielgenau zu gestalten.\cite{appinio_zielgruppenanalyse} \cite{webdesign_journal_zielgruppenanalyse} \\

\textbf{Zielgruppenanalyse als iterativer Prozess}\\
Gerade im digitalen Kontext ist außerdem wichtig, dass Zielgruppen sich verändern. Deshalb ist Zielgruppenanalyse kein einmaliger Schritt, sondern ein iterativer Prozess, der
regelmäßig überprüft und aktualisiert werden sollte.\cite{appinio_zielgruppenanalyse} \cite{webdesign_journal_zielgruppenanalyse} \\

\textbf{Segmentierung von Zielgruppen}\\
Um eine Zielgruppe greifbar zu machen, wird sie anhand verschiedener Kriterien segmentiert.
Je mehr Perspektiven kombiniert werden, desto realistischer wird das Bild der Zielgruppe.\cite{appinio_zielgruppenanalyse} \cite{webdesign_journal_zielgruppenanalyse} \\

\textbf{Geografische, demografische und sozioökonomische Kriterien}\\
Ergänzend dazu spielen geografische Kriterien eine wichtige Rolle, insbesondere bei lokalen oder sprach- und kulturabhängigen Angeboten.
Demografische Merkmale sowie sozioökonomische Faktoren wie Bildung, Beruf oder Einkommen tragen dazu bei, die Zielgruppe weiter zu konkretisieren. \cite{appinio_zielgruppenanalyse} \cite{webdesign_journal_zielgruppenanalyse}\\

Wichtig ist, dass diese Zielgruppen keine Fantasiefiguren sind, sondern auf Daten beruhen. Sie helfen Teams dabei, Entscheidungen zu treffen über themen wie
welche Funktionen Priorität haben, welche Inhalte fehlen oder welcher Kommunikationsstil sich am besten eignet. \cite{appinio_zielgruppenanalyse} \cite{webdesign_journal_zielgruppenanalyse}

\vspace{0.5em}

\textbf{Informationsarchitektur \& Navigation}\\
Die besten Inhalte entfalten nur dann Wirkung, wenn Nutzer sie schnell finden
und einordnen können. Genau hier setzt die Informationsarchitektur an: Sie beschreibt die Strukturierung, 
Gliederung und Verknüpfung von Informationen in digitalen Anwendungen. Ziel ist es, eine 
logische Hierarchie zu schaffen, die Inhalte verständlich organisiert und dadurch die 
Auffindbarkeit sowie die Orientierung verbessert. Informationsarchitektur wirkt meist 
„unsichtbar“, beeinflusst aber unmittelbar, ob eine Website oder eine App als klar und intuitiv 
oder als unübersichtlich wahrgenommen wird. \cite{b13_informationsarchitektur} \cite{eresult_informationsarchitektur}\\

Informationsarchitektur ist das konzeptionelle „Gerüst“ einer Website oder Anwendung. 
Sie umfasst nicht nur die Anordnung von Inhalten innerhalb eines Hauptmenüs, sondern auch das 
„Wie und Wo“: Welche Inhalte und Funktionen existieren, wie sind sie gebündelt, wie hängen 
sie zusammen und an welchen Stellen werden sie angeboten. Damit betrifft Informationsarchitektur
auch die Struktur einzelner Seiten, bezüglich der Platzierung von Modulen wie Menüs, Suchfunktionen,
Newsfeeds oder Profilbereichen. \cite{b13_informationsarchitektur} \cite{eresult_informationsarchitektur}\\

Auf Basis etablierter Informationsarchitektur-Modelle lässt sie sich in vier 
zentrale Komponenten gliedern:\cite{b13_informationsarchitektur} \cite{eresult_informationsarchitektur}
\begin{itemize}
    \item Organisationssysteme: Definieren, wie Inhalte gruppiert und kategorisiert werden (z. B. thematisch, alphabetisch, nach Nutzerrollen).
    \item Navigationssysteme: Bestimmen, wie Nutzer durch die Informationsstruktur geführt werden (z. B. Menüs, Suchfunktionen).
    \item Suchsysteme: Ermöglichen das gezielte Auffinden von Inhalten über Suchfunktionen und Filter.
    \item Kennzeichnungssysteme: Sorgen für klare Beschriftungen und Metadaten, die Inhalte verständlich machen.
\end{itemize}
\vspace{0.5em}
Die Informationsarchitektur wird idealerweise bereits in frühen Projektphasen definiert, sie sorgt für eine
bessere User Experience, indem sie Klarheit schafft und Suchzeiten minimiert. Sie sorgt außerdem für mehr energetische Effizienz,
da der User weniger Klicks braucht um ans Ziel zu kommen, jenes ist vor allem bei Websites relevant. \cite{b13_informationsarchitektur} \cite{eresult_informationsarchitektur}\\

Ein gutes Informationsarchitektur-Design achtet auf folgende Prinzipien:\cite{b13_informationsarchitektur}\cite{eresult_informationsarchitektur}
\begin{itemize}
    \item begrenzte Auswahl pro Ebenen
    \item angemessene Offenlegung von Informationen
    \item Mehrfachklassifikation, mehrere Wege führen zum Ziel
    \item Skalierbarkeit, die Struktur sollte zukünftiges Wachstum und neue Inhalte berücksichtigen. 
\end{itemize}

\vspace{0.5em}

\textbf{Mockups, Prototypen \& Wireframes}\\
Wireframes und Prototypen sind zentrale Bestandteile des modernen Designprozesses für digitale Produkte wie Websites, 
Webanwendungen oder mobile Apps. Sie dienen dazu, Ideen frühzeitig zu strukturieren, zu visualisieren und zu überprüfen, 
bevor zeit- und kostenintensive Entwicklungsarbeiten beginnen. Obwohl die Begriffe im Alltag häufig synonym verwendet werden, 
erfüllen Wireframes und Prototypen unterschiedliche Aufgaben und besitzen jeweils eigene Eigenschaften und Zielsetzungen. \cite{popwebdesign_wire}\cite{justinmind_wireframe_prototyp_mockup}\cite{miro_wire}\\

\textbf{Wireframes: Struktur und Funktion als Grundlage}

Ein Wireframe ist eine vereinfachte, meist statische Darstellung eines digitalen Produkts. Er bildet das grundlegende 
Gerüst einer Anwendung ab und konzentriert sich auf Struktur, Layout und Informationsarchitektur. Im Vordergrund steht 
die Frage, welche Inhalte wo platziert werden und wie Nutzer durch das System geführt werden. Visuelle Details wie Farben, 
Typografie oder Bilder spielen dabei eine untergeordnete Rolle oder fehlen vollständig. Häufig werden Wireframes in 
Graustufen mit einfachen Boxen, Linien und Platzhaltern umgesetzt. \\

Wireframes werden vor allem in frühen Phasen des Designprozesses eingesetzt. Sie ermöglichen es, Konzepte schnell zu 
erstellen, zu diskutieren und zu verändern. Dadurch eignen sie sich besonders gut, um erste Ideen mit Stakeholdern 
abzustimmen und grundlegende Designentscheidungen zu treffen. Ein weiterer Vorteil liegt in der Kosten- und Zeiteffizienz, 
da Anpassungen ohne großen Aufwand vorgenommen werden können. Gleichzeitig liegt hier auch eine Einschränkung: Da Wireframes 
kaum Interaktivität und keine realistischen Inhalte enthalten, sind sie nur bedingt geeignet, um Benutzerfreundlichkeit oder 
Nutzerinteraktionen zu testen.\cite{popwebdesign_wire}\cite{justinmind_wireframe_prototyp_mockup}\cite{miro_wire}
\vspace{0.5em}

\textbf{Prototypen: Interaktion und Realitätsnähe} \\
Prototypen bauen auf Wireframes auf und stellen eine weiterentwickelte, interaktive Version des Produkts dar. 
Sie reichen von einfachen Low-Fidelity-Prototypen bis hin zu High-Fidelity-Prototypen, die dem späteren Endprodukt 
visuell und funktional sehr nahekommen. Prototypen enthalten reale Inhalte, UI-Elemente, Animationen und definierte 
Interaktionen. Nutzer können klicken, navigieren und Abläufe realistisch nachvollziehen.\cite{popwebdesign_wire}\cite{justinmind_wireframe_prototyp_mockup}\cite{miro_wire} \\

Der Hauptzweck von Prototypen liegt in der Validierung von Designentscheidungen. Durch Nutzertests 
lassen sich Missverständnisse oder Schwächen im Benutzerfluss frühzeitig erkennen. Dadurch kann wertvolles Feedback 
gesammelt werden, bevor mit der technischen Umsetzung begonnen wird. Zwar ist die Erstellung von Prototypen aufwendiger 
als die von Wireframes, langfristig helfen sie jedoch, Fehlentwicklungen und teure Nachbesserungen zu vermeiden.\cite{popwebdesign_wire}\cite{justinmind_wireframe_prototyp_mockup}\cite{miro_wire} \\

\textbf{Mockups: Visueller Designprozess} \\
Neben Wireframes und Prototypen werden häufig auch Mockups verwendet. Mockups sind statische, visuell ausgearbeitete 
Darstellungen eines Produkts. Sie zeigen Farben, Typografie und das visuelle Erscheinungsbild sehr genau, bieten 
jedoch keine Interaktivität. Während Wireframes die Struktur festlegen und Prototypen die Nutzung simulieren, 
dienen Mockups vor allem der Beurteilung des visuellen Designs.\cite{popwebdesign_wire}\cite{justinmind_wireframe_prototyp_mockup}\cite{miro_wire}\\

\textbf{Zusammenspiel im Designprozess} \\
Die verschiedenen Design-Artefakte sind keine konkurrierenden Phasen, sondern ergänzen sich. In der Regel beginnt der
Designprozess mit Wireframes, die als Fundament dienen. Danach werden Mockups angefertigt um das visuelle Erscheinungsbild zu evaluieren. Sobald Struktur und Funktionalität festgelegt sind, 
werden diese Entwürfe in Prototypen überführt, um das Nutzungserlebnis realistisch darzustellen und zu testen. 
In manchen Fällen entstehen auch High-Fidelity-Wireframes, die bereits detaillierter sind, jedoch noch nicht den 
vollen Funktionsumfang eines Prototyps besitzen. \cite{popwebdesign_wire}\cite{justinmind_wireframe_prototyp_mockup}\cite{miro_wire}
\vspace{0.5em}

\textbf{Visuelles Design}\\
Visuelles Design bezeichnet die gezielte Gestaltung visueller Elemente mit dem Ziel, digitale Produkte 
ästhetisch ansprechend und zugleich funktional zu gestalten. Es bildet eine zentrale Schnittstelle zwischen
Gestaltung und Nutzererfahrung, da es nicht nur das äußere Erscheinungsbild eines Produkts prägt, sondern 
auch maßgeblich beeinflusst, wie Inhalte wahrgenommen, verstanden und genutzt werden. In der digitalen Welt 
ist visuelles Design daher ein wesentlicher Bestandteil von Webanwendungen, Software, Marketingplattformen 
und interaktiven Medien.\cite{Toolmaster_Design}\cite{StudySmarter_Visual_Design}\\

\textbf{Bestandteile des visuellen Designs}\\
Im Kern umfasst visuelles Design alle sichtbaren Gestaltungskomponenten eines digitalen Produkts. Dazu 
zählen Bilder, grafische Elemente, Farben, Typografie, Layoutstrukturen sowie der gezielte Einsatz von 
Weißraum. Diese Elemente wirken nicht isoliert, sondern entfalten ihre Wirkung im Zusammenspiel. Ein
konsistentes und durchdachtes visuelles Design trägt zur Stärkung der Markenidentität bei, erhöht die 
Wiedererkennbarkeit und unterstützt eine klare Kommunikation von Inhalten.\cite{Toolmaster_Design}\cite{StudySmarter_Visual_Design}\\

\textbf{Zentrale Gestaltungselemente}\\
Besondere Bedeutung kommt den einzelnen Gestaltungselementen zu. Bilder und grafische Elemente dienen nicht 
nur der Dekoration, sondern übernehmen eine informative und emotionale Funktion. Sie können komplexe Inhalte 
vereinfachen, Aufmerksamkeit lenken und eine Verbindung zum Nutzer herstellen. Typografie beeinflusst maßgeblich 
die Lesbarkeit und visuelle Hierarchie. Schriftart, -größe und -gewicht bestimmen, welche Inhalte priorisiert 
wahrgenommen werden. Die Farbgestaltung wirkt stark auf die emotionale Ebene und kann Stimmungen erzeugen, 
Kontraste verstärken sowie Orientierung bieten.\cite{Toolmaster_Design}\cite{StudySmarter_Visual_Design}\\

\textbf{Evaluation und Optimierung im Designprozess}\\
Moderne visuelle Designs werden zunehmend daten- und nutzerorientiert entwickelt. Mithilfe von 
Benutzeranalysen, Usability-Tests und Rapid-Prototyping-Methoden lassen sich Designentscheidungen 
überprüfen und optimieren. Die Wirkung visueller Gestaltung ist somit messbar und kann gezielt zur 
Verbesserung der Nutzererfahrung und Markenwahrnehmung eingesetzt werden.\cite{Toolmaster_Design}\cite{StudySmarter_Visual_Design}

% ==================================================
\section{Monetarisierung von mobilen Apps}

% --------------------------------------------------
\subsection{Grundlagen \& Marktüberblick}

\textbf{Entwicklung des mobilen App-Marktes}\\
Der mobile App-Markt wächst kontinuierlich und bietet vielfältige Monetarisierungsmöglichkeiten.

\textbf{App-Ökosysteme}\\
Die wichtigsten Distributionsplattformen für mobile Apps sind:
\begin{itemize}
    \item Apple App Store
    \item Google Play Store
\end{itemize}

\textbf{Nutzerverhalten \& Marktstatistiken}\\
Marktanalysen zeigen, dass Nutzer zunehmend bereit sind, für digitale Inhalte und Zusatzfunktionen zu bezahlen.

\textbf{Geschäftsmodelle digitaler Produkte}\\
Digitale Produkte ermöglichen unterschiedliche Erlösmodelle wie Einmalkäufe, Abonnements oder In-App-Käufe.

% --------------------------------------------------
\subsection{Werbung als Einnahmequelle}

\textbf{Arten von Werbung in mobilen Apps}\\
In mobilen Apps kommen verschiedene Werbeformen wie Banner-, Interstitial- oder Videoanzeigen zum Einsatz.

\textbf{Implementierung von Werbung}\\
Werbung kann über externe Werbenetzwerke technisch einfach in Apps integriert werden.

\textbf{Werbenetzwerke}\\
Häufig genutzte Werbenetzwerke sind:
\begin{itemize}
    \item Google AdMob
    \item Smaato
    \item Meta Audience Network
    \item AppLovin
\end{itemize}

\textbf{Auswirkungen auf die User Experience}\\
Eine ausgewogene Platzierung von Werbung ist entscheidend, um Einnahmen zu erzielen, ohne die Nutzererfahrung negativ zu beeinflussen.

% --------------------------------------------------
\subsection{Sponsoring \& Partnerschaften}

\textbf{Grundprinzip von App-Sponsoring}\\
Unternehmen können als Sponsoren innerhalb der App auftreten und so gezielt Zielgruppen erreichen.

\textbf{Kooperationen mit Tourismusverbänden}\\
Tourismusverbände eignen sich besonders für regionale oder themenspezifische Anwendungen.

\textbf{Hotel- \& Buchungsplattformen}\\
Mögliche Kooperationspartner sind:
\begin{itemize}
    \item Trivago
    \item Check24
    \item Ab-in-den-Urlaub
\end{itemize}

\textbf{Wirtschaftliches Potenzial}\\
Partnerschaften können eine langfristige und stabile Einnahmequelle darstellen.

% --------------------------------------------------
\subsection{Kostenpflichtige App-Modelle}

\textbf{Einmaliger Kauf}\\
Die App wird gegen eine einmalige Gebühr angeboten.

\textbf{Abonnement-Modelle}\\
Nutzer zahlen regelmäßig für den Zugriff auf zusätzliche Inhalte oder Funktionen.

\textbf{Zahlungsbereitschaft der Nutzer}\\
Die Zahlungsbereitschaft hängt stark vom wahrgenommenen Mehrwert der App ab.

% --------------------------------------------------
\subsection{In-App-Käufe}

\textbf{Grundlagen von In-App-Purchases}\\
Zusätzliche Inhalte oder Funktionen können direkt innerhalb der App erworben werden.

\textbf{Personalisierung}\\
Beispiele für personalisierte Inhalte sind:
\begin{itemize}
    \item Profilbilder
    \item Banner
\end{itemize}

\textbf{Premium-Funktionen}\\
Bestimmte Funktionen sind ausschließlich zahlenden Nutzern vorbehalten.

\textbf{Abo-Modelle}\\
Monatliche Abonnements ermöglichen den Zugang zu erweiterten Features.

% --------------------------------------------------
\subsection{Spenden \& freiwillige Unterstützung}

\textbf{Spendenmodelle in Apps}\\
Nutzer haben die Möglichkeit, die App freiwillig finanziell zu unterstützen.

\textbf{Transparenz \& Vertrauen}\\
Eine offene Kommunikation über die Verwendung der Spenden stärkt das Vertrauen der Nutzer.

% --------------------------------------------------
\subsection{Vergleich \& Bewertung der Monetarisierungsstrategien}

\textbf{Wirtschaftliches Potenzial}\\
Die Monetarisierungsstrategien unterscheiden sich deutlich hinsichtlich ihres Ertragspotenzials.

\textbf{Nutzerakzeptanz}\\
Die Akzeptanz der Nutzer ist ein zentraler Erfolgsfaktor für jedes Monetarisierungsmodell.

\textbf{Geeignete Strategie für die entwickelte App}\\
Abschließend wird die am besten geeignete Monetarisierungsstrategie für die entwickelte App bewertet.
