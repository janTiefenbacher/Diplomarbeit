
%% Akronyme, die im Text verwendet werden
\newacronym{apis}{APIs}{Application Programming Interfaces}

%% Glossar
\newglossaryentry{gps}
{
        name=GPS,
        description={Global Positioning System, ein satellitengestütztes Navigationssystem zur Bestimmung von Position, Geschwindigkeit und Zeit auf der Erde}
}
\newglossaryentry{es6}
{
        name=ES6-Features,
        description={Erweiterungen von ECMAScript 6 (ES6), die moderne JavaScript-Funktionen wie Klassen, Module, Arrow Functions und verbesserte Variablendeklarationen bereitstellen}
}
\newglossaryentry{erd}
{
        name=Entity-Relationship-Diagramm (ERD),
        description={Grafische Darstellung eines Datenbankmodells, die Entitäten, deren Attribute sowie die Beziehungen zwischen ihnen zeigt. Es dient der Planung und Strukturierung relationaler Datenbanken}
}
\newglossaryentry{publish-subscribe}{
    name={Publish-Subscribe-Modell},
    description={Ein Kommunikationsmuster, bei dem Nachrichten von einem Sender (Publisher) an einen zentralen Kanal gesendet werden, ohne die Empfänger direkt zu kennen. Interessierte Empfänger (Subscriber) abonnieren diesen Kanal und erhalten relevante Ereignisse automatisch. Dieses Modell ermöglicht eine lose Kopplung, skalierbare Echtzeitkommunikation und effiziente Synchronisation zwischen mehreren Systemkomponenten}
}
\newglossaryentry{ui}
{
    name=UI,
    description={User Interface (Benutzeroberfläche). Bezeichnet die sichtbaren und interaktiven Elemente einer Anwendung, über die Nutzer mit dem System interagieren}
}

\newglossaryentry{css}
{
    name=CSS,
    description={Cascading Style Sheets. Stylesheet-Sprache zur Gestaltung von Layout, Farben, Abständen und visuellen Eigenschaften von Benutzeroberflächen in Webanwendungen}
}
\newglossaryentry{overlay}
{
    name=Overlay,
    description={Grafisches Element, das über anderen Inhalten angezeigt wird, ohne deren Struktur zu verändern, z.\,B. Marker, Labels oder UI-Komponenten auf einer Karte}
}
