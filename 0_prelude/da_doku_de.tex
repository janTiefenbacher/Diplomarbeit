% DIPOLOMARBEITS TABELLEN DEUTSCH ########################
\newpage
\begin{center}
\textbf{\LARGE DIPLOMARBEIT}

\textbf{Dokumentation}
\end{center}

%Schülernamen mit Klasse oder Schuljahr
\begin{tikzpicture}
\node[da_dokubox] {
    \textcolor{darkgray}{Verfasser*innen} \\ 
    \begin{itemize} [itemsep=0pt, topsep=0pt]
        \item[] Jan Tiefenbacher, 5BHITM
        \item[] Laurenz Pichler, 5BHITM
    \end{itemize}
    \textcolor{darkgray}{Abteilung} \\ 
    \begin{itemize} [itemsep=0pt, topsep=0pt]
        \item[] Informationstechnologie
        \item[] Ausbildungsschwerpunkt: Medientechnik
    \end{itemize}
    \textcolor{darkgray}{Schuljahr} \\ 
    \begin{itemize} [itemsep=0pt, topsep=0pt]
        \item[] 2025/26
    \end{itemize}
        \textcolor{darkgray}{Thema der Diplomarbeit} \\ 
    \begin{itemize} [itemsep=0pt, topsep=0pt]
        \item[] Wosamma geoinformationsbasiertes Quizzspiel
    \end{itemize}
        \textcolor{darkgray}{Kooperationspartner} \\ 
    \begin{itemize} [itemsep=0pt, topsep=0pt]
        \item[] Xinger Solutions GmbH
    \end{itemize}
}; %node (Rahmen)
\end{tikzpicture}

\vspace{0.5cm}

\begin{tikzpicture}
\node[da_dokubox] {
    \textcolor{darkgray}{Aufgabenstellung} \\ 
Ziel des Projekts WoSamma war die Entwicklung einer mobilen, geoinformationsbasierten Quiz-App, die sich auf Österreich fokussiert. Spieler sollen anhand von Street-View-Bildern erraten, an welchem Ort sie sich befinden, und dafür Punkte erhalten. Die App soll sowohl Lern- als auch Unterhaltungszwecke erfüllen und österreichisches Geografiewissen spielerisch vermitteln. Neben der Einzelspieler-Funktion war auch die Umsetzung von Mehrspielermodi, Ranglisten, Freundesystemen und einem Administrationsbereich geplant. Die Anwendung sollte plattformunabhängig lauffähig sein und insbesondere auf mobilen Endgeräten eine flüssige Nutzererfahrung bieten.

    }; %node (Rahmen)
\end{tikzpicture}

\vspace{0.5cm}

\begin{tikzpicture}
\node[da_dokubox, align=left, text width=10cm] {
    \textcolor{darkgray}{\textbf{Realisierung}}\\[4pt]
    \begin{itemize}[leftmargin=1em]
        \item \textbf{Programmiersprache / Framework:} React Native mit TypeScript
        \item \textbf{Backend \& Datenbank:} Supabase (PostgreSQL, Auth, Storage)
        \item \textbf{Deployment:} Mit Apple Developer Konto auf dem Handy nutzbar
        \item \textbf{UI/UX-Design:} Figma (Mockups \& Komponenten)
        \item \textbf{Versionsverwaltung:} Git \& GitHub
        \item \textbf{Projektmanagement:} Jira (agiles Vorgehen, Sprints)
        \item \textbf{Besonderheiten bei der Entwicklung:}
        \begin{itemize}
            \item Performance-Optimierung für mobile Geräte
            \item API-Sicherheit \& Authentifizierung
            \item Plattformübergreifende Kompatibilität (iOS \& Android)
            \item Echtzeitfunktionen (Chat, Multiplayer)
        \end{itemize}
    \end{itemize}
};
\end{tikzpicture}


\vspace{0.5cm}

\begin{tikzpicture}
\node[da_dokubox] {
    \textcolor{darkgray}{Ergebnisse} \\ 
Eine voll funktionsfähige Mobile-App mit vielfältigen Spielmodi, integriertem Freundesystem und Echtzeit-Chat für ein dynamisches und interaktives Spielerlebnis.

}; %node (Rahmen)
\end{tikzpicture}

\vspace{0.5cm}


