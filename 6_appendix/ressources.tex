\section{Datenträgerbeschreibung}
 
Der abgegebene Datenträger enthält sämtliche zur Diplomarbeit gehörenden Unterlagen, Dokumentationen, Projektdateien sowie ergänzende Materialien.  
Die Daten sind strukturiert in thematisch getrennten Ordnern abgelegt, um eine übersichtliche Nachvollziehbarkeit und eine einfache Auffindbarkeit zu gewährleisten.
 
\subsection*{1. Ordner „Diplomarbeit\_LateX“}
 
Dieser Ordner enthält die vollständigen Quelldateien der schriftlichen Diplomarbeit.
 
\textbf{Inhalt:}
\begin{itemize}
    \item Hauptdokument der Arbeit (.tex-Datei)
    \item Kapiteldateien
    \item Literaturverzeichnis (Bib\TeX-Dateien)
    \item Abbildungen und Grafiken
    \item Konfigurations- und Paketdateien
    \item Kompilierte Endversion der Diplomarbeit im PDF-Format
\end{itemize}
 
Damit ist die vollständige Reproduzierbarkeit der schriftlichen Arbeit gewährleistet.
 
\subsection*{2. Ordner „Figma“}
 
Der Ordner „Figma“ beinhaltet sämtliche im Rahmen der Projektentwicklung erstellten Design- und Prototyping-Dateien.
 
\textbf{Inhalt:}
\begin{itemize}
    \item Benutzeroberflächen-Entwürfe
    \item Wireframes
    \item Mockups
    \item Exportierte Designgrafiken
\end{itemize}
 
\subsection*{3. Datei „Wosamma.zip“}
 
Bei der Datei „Wosamma.zip“ handelt es sich um ein komprimiertes Archiv des entwickelten Softwareprojekts.
 
\textbf{Inhalt:}
\begin{itemize}
    \item Vollständiger Quellcode
    \item Projektstruktur
    \item Ressourcen und Assets
    \item Konfigurationsdateien
    \item Dokumentation bzw. Installationsanleitung (sofern vorhanden)
\end{itemize}
 
Das ZIP-Archiv dient der kompakten Weitergabe und Archivierung des gesamten Projekts.
 
\subsection*{4. Ordner „Pflichtenheft“}
 
Dieser Ordner enthält das im Zuge der Diplomarbeit erstellte Pflichtenheft.
 
\textbf{Inhalt:}
\begin{itemize}
    \item Finale Version des Pflichtenhefts (PDF)
\end{itemize}
 
Das Pflichtenheft dokumentiert die funktionalen und nicht-funktionalen Anforderungen des Projekts.
 
\subsection*{5. Ordner „Video“}
 
Der Ordner „Video“ enthält eine vollständige Videoaufzeichnung des Projekts.  
In diesem Video wird die entwickelte Anwendung einmal vollständig durchlaufen und sämtliche implementierten Funktionen werden demonstriert.
 
\section{Installationsanleitung}
 
Zur Kompilierung und Ausführung des Projekts auf einem physischen iPhone-Gerät wird eine macOS-Umgebung mit installierter Entwicklungsumgebung Xcode benötigt.
 
\subsection*{Voraussetzungen}
 
\begin{itemize}
    \item Apple-Computer mit macOS
    \item Installiertes Xcode (aktuelle Version empfohlen)
    \item Aktive Apple Developer Lizenz
    \item iPhone mit kompatibler iOS-Version
    \item USB-Kabel oder drahtlose Entwicklungsverbindung
\end{itemize}
 
\subsection*{Vorgehensweise}
 
\begin{enumerate}
    \item Öffnen des Projekts in Xcode.
    \item Anmeldung mit einem Apple-Developer-Account in Xcode.
    \item Auswahl des gewünschten Zielgeräts (iPhone).
    \item Gegebenenfalls Anpassen der Signing- und Team-Einstellungen im Projekt.
    \item Kompilieren und Builden des Projekts direkt auf dem angeschlossenen Gerät.
\end{enumerate}
 
Für die Installation auf einem physischen Gerät ist eine gültige Apple Developer Lizenz erforderlich, da nur damit eine Code-Signierung für die Ausführung auf iOS-Geräten möglich ist. Ohne kostenpflichtigen Developer-Account ist lediglich ein eingeschränkter Testbetrieb möglich.
 
Alternativ kann das Projekt auch im iOS-Simulator innerhalb von Xcode ausgeführt werden, wofür keine kostenpflichtige Entwicklerlizenz notwendig ist.